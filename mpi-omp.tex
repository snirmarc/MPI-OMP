	\section{MPI + OpenMP parallelism}
	\subsection{Current State}
	In current implementations, MPI threads are identified with OpenMP threads. 
	If the MPI process executes a multithreaded MPI thread then MPI must 
	executed in mode \texttt{MPI\_THREAD\_FUNNELED} or higher. In the  
	\texttt{MPI\_THREAD\_FUNNELED} mode all MPI calls should be 
	executed on the master thread (thread zero). MPI+OpenMP codes are often 
	written so that MPI calls are executed in sections where the OpenMP code 
	executes sequentially, so that the execution consists of an alternation of 
	parallel compute and sequential 
	communicate phases. This is not necessary: Communication can be concurrent 
	with computation, as long as it executes on the master 
	thread. (Synchronization, such as a barrier, is still needed to ensure 
	proper order between communication calls and code that reads or writes 
	communication buffers.)
	
	The use of concurrent MPI communications can be advantageous when a node 
	has multiple adapters, or when the node code consists of multiple, mostly 
	independent executions (as is the case when a code using multiple MPI 
	processes per node is replaced by a code using multiple threads; see, e.g., 
	\cite{akhmetova2017performance}). If communications can be separated into 
	independent 
	streams, then concurrent MPI communications can be supported simply by 
	running multiple MPI processes on the same node. On nodes with multiple 
	adapters, vendors will provide the ability to bound different processes to 
	distinct adapters; single adapters typically support multiple independent 
	windows (independent "virtual adapters"). MPI 3 provides support for the 
	sharing of memory segments by MPI processes running on the same node 
	(\texttt{MPI\_WIN\_ALLOCATE\_SHARED}).  Node code will consist of multiple 
	processes each running an OpenMP program. Communication between processes 
	on the same node can either use MPI calls or a shared segment. The 
	disadvantage of this approach is the need for two levels of shared-memory 
	parallelism (interprocess and interthread), the static allocation of 
	resources to the first level, and the lack of support in C, C++ or Fortran 
	for two different types of memory.
	
	In order to support concurrent MPI calls from one OpenMP program, one will 
	need to use \texttt{MPI\_THREAD\_MULTIPLE}. With it, any thread can call 
	MPI. If a thread executes a blocking call (blocking Send or Receive or 
	Wait), the calling thread stops executing, but the other threads will 
	continue their execution. In such a case, it is desirable to use work 
	sharing constructs (parallel loops with a dynamic schedule or untied 
	tasks), 
	in order to make sure that the threads that did not call MPI, or completed 
	their blocking MPI call earlier, pick the load from the threads blocked in 
	MPI calls. Latency per message will be slightly higher, 
	assuming 
	concurrent accesses are rare, it can be significantly higher, if the number 
	of pending communications per rank grows into the 100's or if many MPI 
	calls occur simultaneously.
	
\subsection{Future}
The MPICH and the OpenMPI teams are working to improve the performance of MPI 
with multithreading. One can expect the overhead of 
\texttt{MPI\_THREAD\_MULTIPLE} to decrease in future MPI releases. One can also 
expect performance improvements in recent MPI 3 constructs, such as the new 
functions for one-sided communication, nonblocking collectives and sparse 
collectives. One-sided communication has the potential to reduce MPI overheads 
and should be considered, in replacement to two-sided communication, as 
implementations improve. 

The use of a dedicated progress thread will ensure more consistency in the 
performance of nonblocking communication calls, and increased communication to 
computation overlap. It should be considered for any application that  requires 
communication/computation overlap.


If MPI 4 will support multiple endpoints per process, then it will be possible 
to leverage efficiently multiple NICs per process, and to reduce the thrashing 
that occurs when many threads call MPI simultaneously: Communication using 
distinct endpoints do not conflict with each other.

An MPI library can reduce the overhead for matching sends and receives if it is 
known  that wildcard receives (\texttt{MPI\_ANY\_SOURCE} and 
\texttt{MPI\_ANY\_TAG}) are not mixed with regular receives for the same 
communicator. It also becomes much easier to support 
the matching in hardware. The use of matching lengths for send and receive 
buffers can save one of the current three message exchanges that are needed in 
the rendez-vous protocol. If MPI 4 will support the additional communicator 
info keys (\#11, \#58) and implementations will take advantage of these 
assertions, 
then receive performance will not degrade even when there are many pending 
receives or sends and the rendezvous protocol will have lower latency. Users 
can prepare their codes for these changes by avoiding 
the mixing of wild-card receives with regular receives on the same communicator 
and (less importantly) by matching send buffer and receive buffer lengths. To 
the extent that 
current exascale benchmarks are representative, this restrictions will not 
affect many codes \cite{klenk2017overview}.

The performance of derived datatypes is unlikely to improve much. Basically, 
a communication that uses a derived datatype has to interpret the datatype, 
while 
user-defined pack or unpack code is compiled; compilation beats runtime 
interpretation. Also, the inplementation of derived datatypes in current MPI 
libraries seems less than optimal \cite{carpen2017expected}. The situation 
could 
be remedied by 
using a Just-In-Time 
compiler (JIT) but this is difficult to distribute as part of a library. A 
refactoring tool that replaces communication using derived datatypes with 
pack/unpack routines followed by communication using a contiguous buffer, 
would be useful. The benefit can be very significant for communication buffers 
on GPUs.

